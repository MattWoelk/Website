%This tex file is based off of Asgn1-2009.tex from the course webpage
\documentclass[a4paper,12pt]{article}

% Declaration section

% Use \usepackage[options]{fullpage} to set all margins to 1.5 cm.
\usepackage{fullpage}
% Set up environment for displaying typewriter characters without special meaning
\usepackage{alltt}
% Use to insert images into your document.
\usepackage{graphicx}

% Use hyperref package to set up hyperlinks for webpages as well as document references, sections, TOC.
\usepackage{hyperref}

\hypersetup{linktocpage=true,colorlinks=true,linkcolor=blue,citecolor=blue,pdfpagemode=UseNone,pdfstartview={XYZ 1000 1000 1}}

\title{Assignment One}
\author{Matthew Woelk, ID 6840262\\ \href{mailto:umwoelk@cc.umanitoba.ca}{umwoelk@cc.umanitoba.ca}}



% Begin document environment
\begin{document}

% Set up title page.  Notice that the title page has no page number.
\maketitle
\thispagestyle{empty}

% Set up TOC for title page.
\tableofcontents

% Start a new page after the TOC.
\newpage

% Begin absract environment.
\begin{abstract} 
This report provides a brief introduction to Systems Engineering.  It takes a look at a few of the main organizations on the topic including IEEE, IET, and INCOSE.  Differing definitions of the topic are also be discussed, and an example of an embedded system is provided.\\

{\bf Keywords}: IEEE, IET, INCOSE, System Engineering, SE. 

% Exit from abstract environment.
\end{abstract}


\section{What is System Engineering?}
This section of the report focuses on an answer to question What is system engineering?  Principal sources of information about SE which are used in this report are INCOSE-UK~\cite{INCOSE-UK}, INCOSE-US~\cite{INCOSE-US}, IET~\cite{IET}, and IEEE~\cite{IEEE}.

Throughout this report, information is given about the differing descriptions, and different organizations that surround the idea of SE. There is no universally accepted definition for SE, but the general idea is consistent, and a definition is given in this report.

\section{Topical Problems}
The main topic in this report is system engineering.  The problems (questions) related to this topic are given in the following subsections.

\subsection{What is System Engineering?}

According to Wikipedia.org, Systems Engineering is "an interdisciplinary field of engineering that focuses on how complex engineering projects should be designed and managed."\cite{wiki}  

An example of an embedded, real-time engineering system would be the wireless electronic stethoscope created by a team from Calvin College.  This stethoscope streams, filters, records, stores, and plays audio data from a medical patient\cite{steth}.  A block diagram of this system is contained in figure~\ref{fig1}.
\begin{center}
\begin{figure}[ht]
	\begin{center}
		\caption{Electronic stethoscope block diagram}
		\includegraphics[width=13cm]{blockdiagram.png}
	\end{center}
	\label{fig1}
\end{figure}
\end{center}

\subsection{INCOSE}
INCOSE stands for International Council on Systems Engineering.  There are many INCOSE organizations, UK-based~\cite{INCOSE-UK}, US-based~\cite{INCOSE-US} plus others.   There are six INCOSE regions~\cite{regions}.  The The North Star Chapter of INCOSE follows the main INCOSE website's definition of SE, which says that SE is the discipline of developing systems product or process, start to finish, while considering operations, performance, testing, manufacturing, cost, schedule, training, support, and disposal.\cite{SEns}\cite{SEdef}  The UK Chapter of INCOSE provides a more lose description of SE, which is that it is "Big Picture thinking, and the application of Common Sense to projects," and "a structured and auditable approach to identifying requirements, managing interfaces and controlling risks throughout the project lifecycle."\cite{UKdef}  The main features of the UK and US INCOSE websites are very similar. They include information about tutorials that are taking place, newsletters that are available, and general information about what their purpose is. 

\subsection{IET}
IET~\cite{IET} stands for Institution of Engineering and Technology.  The IET was formed in 2006, and has since then grown to have over 150 000 members throughout 37 countries.  The IET produces a bibliographic information service for scientific and technical literature called Inspec~\cite{Inspec}.  The IET's mission is "to build an open, flexible and limitless global knowledge network supported by individuals, companies and institutions and facilitated by the IET and its members."\cite{IET-Mission}.

\subsection{IEEE}
IEEE~\cite{IEEE} stands for Institute of Electrical and Electronics Engineers.  Information about the IEEE higher speed 802.11 wireless local area network standards is located on the IEEE standards website\cite{802.11}.  Information about the IEEE {\bf S}ystems, {\bf M}an, \& {\bf C}ybernetics Society is located at \url{http://www.ieeesmc.org/}\cite{ieeesmc}  IEEE SMC's basic mission is to promote the theory, practice, and interdisciplinary aspects of systems science and engineering, human-machine systems, and cybernetics through conferences, publications, and other activities.  IEEE SMC is significant because it deals directly with the entire spectrum of problems that confront system engineering.



\newpage
\addcontentsline{toc}{section}{References}
\begin{thebibliography}{100}
	
\bibitem{wiki} Systems Engineering at Wikipedia.org: \url{http://en.wikipedia.org/wiki/Systems_engineering}

\bibitem{INCOSE-UK} INCOSE-UK: \url{http://www.incose.org.uk/}

\bibitem{INCOSE-US} INCOSE-US plus others: \url{http://www.incose.org/chapters/websites.aspx}

\bibitem{regions}: INCOSE geographic regions:\\
\url{http://www.incose.org/chapters/geographic.aspx}

\bibitem{SEns} North Star Chapter SE definition: \url{http://www.incose.org/northstar/images/Trifold%20Flyer%20Contents.doc}

\bibitem{SEdef} INCOSE SE definition: \url{http://www.incose.org/practice/whatissystemseng.aspx}

\bibitem{UKdef} INCOSE UK definition of SE: \url{http://www.incose.org.uk/Downloads/z4-Leafletv0_1.doc}

\bibitem{steth} Calvin College electronic stethoscope project: \url{http://knightvision.calvin.edu/bbcswebdav/orgs/ENGR/senior-projects/2007-08/Team06/downloads/T06_Final_Report.pdf}

\bibitem{IET} IET: \url{http://www.theiet.org/about/today/index.cfm}

\bibitem{IEEE} IEEE: \url{http://www.ieee.org/portal/site}

\bibitem{Inspec} Inspec: \url{http://www.theiet.org/publishing/inspec/index.cfm}

\bibitem{IET-Mission} IET Mission: \url{http://www.theiet.org/about/today/vision/index.cfm}

\bibitem{802.11} IEEE 802.11 standard: \url{http://standards.ieee.org/getieee802/802.11.html}

\bibitem{ieeesmc} IEEE SMC \url{http://www.ieeesmc.org/}

\end{thebibliography}



\section*{Matthew Woelk}
\addcontentsline{toc}{section}{Brief Biography}
$\put(0,0){\parbox{.7\linewidth}{
Matthew Woelk was born in Winnipeg, Manitoba, which increased his family's size by 33\%.   His high-school years were spent playing music, being in dramas, and being co-president at his school.  He now is taking studies at the University of Manitoba to become a certified Computer Engineer.  Most of Matthew's time outside of school is spent being a youth leader, an improv coach, in a ska band, and spent making little programs in any computer language he can get his hands on. (He says that the purpose of these computer programs is to help him accomplish tasks more quickly, but we all know that he does it mostly for the fun of it.)  Matt doesn't know where he wants to end up in the future, but he knows that finding out is going to be quite a ride.}
}$
$\put(330,-100){\includegraphics[width=5cm]{pose.jpg}}$

\end{document}
%-----------------------------------------------------